Este trabalho de conclusão reporta o desenvolvimento de um \emph{parser} para a lingua portuguesa utilizando as técnicas estatísticas de processamento de linguagem natural desenvolvidas por Michael Collins, utilizando a ferramenta implementado por Dan Bikel, que tornou possível a parametrização e extensão das suas bibliotecas para possíveis adaptações do código. O trabalho desenvolveu-se em torno das seguintes etapas:

\begin{itemize}
	\item Estudar as técnicas envolvidas no desenvolvimento de um \emph{parser}, em particular os modelos de \emph{parsing} de Collins.
	\item Dominar o uso da ferramenta de \emph{parsing} de Bikel.
	\item Fazer um estudo detalhado dos parâmetros de implementação de Bikel, pois a eficácia dos algoritmos depende fundamentalmente dos ajustes desses parâmetros.
	\item Utilizar a ferramenta de Bikel para construção de um \emph{parser} para a língua portuguesa. O treinamento do \emph{parser} foi feito utilizando-se o \emph{corpus} anotado Floresta Sintática (Projeto Linguateca).
	\item Desenvolver módulos de pré e pós processamento do \emph{corpus}.
\end{itemize}

