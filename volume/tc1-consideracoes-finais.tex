
A escolha de um bom \emph{tagset} é fundamental para o sucesso de um \emph{parser}. Um bom \emph{tagset} para um \emph{parser} é aquele que possui uma boa caraterística de ``equivalência distribucional" em termos sintáticos; isto é, palavras que ocorrem tipicamente nas mesmas posições nas sentenças têm mesmo POS, enquanto que as que têm características de distribuição diferentes na mesma sentença têm POS diferente. 

Do ponto de vista experimental, obtivemos resultados de treinamento razoáveis, cuja qualidade parece bastante dependente de uma anotação bem feita. 

Um dos principais motivos de incremento nos resultados com relação a trabalhos da literatura, foi mais uma vez, com relação a construção das \emph{head-find rules} para os constituintes encontrados no corpus utilizado. O \emph{parser} de Bikel que implementa o modelo gerativo baseado na noção \emph{head-centering}, tem como principal ferramenta no processo o uso de \emph{head-find rules}, que se baseia no núcleo dos sintagmas que é o elemento principal de todo o processo de geração.

O \emph{Parser} de Bikel, além de ser possível emular os modelos de Collins, também possibilita diferentes parametrizações quanto a algoritmos utilizados e implementação de regras de \emph{head-find}, por exemplo. Alguns parâmetros melhoram a performance quanto a acertos no momento de análise de sentenças mas levam um tempo maior para finalizar, já outras configurações permitem que o tempo de análise das sentenças seja mais rápido porém a qualidade do resultado pode diminuir.

Outro fator de grande influência no incremento dos resultados foi a utilização de lematização das palavras do corpus, mais especificamente dos verbos, experimento este não abordado pelos trabalhos anteriores na literatura, acreditamos que a lematização contribuiu no processo de aprendizado do \emph{parser}.

Como trabalho futuro, sugerimos uma análise de um ponto de vista estrutural das regras implementadas por Bikel em seu \emph{parser}, pois percebemos que existem algumas regras, provavelmente de otimização, que estão \emph{hard coded} no código fonte que são dependentes do corpus utilizado originalmente (Penn TreeBank). 

Também observando a matriz de confusão apresentada nos experimentos realizados, pode-se tentar melhorar os resultados obtidos, identificando os maiores erros e possibilitando focar especificamente nesses casos. 

Também ajustes nos parâmetros que o \emph{parser} necessita para tratar as palavras desconhecidas podem levar a melhor ganho na performace.
