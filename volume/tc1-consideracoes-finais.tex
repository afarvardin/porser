Percebemos que o critério de anotação utilizada em um corpus tem grande influência nos resultados. Verificamos uma melhora expressiva com relação aos primeiros testes realizados após a substituição dos caracteres ``{-}'' por ``\_''.

Do ponto de vista experimental, obtivemos resultados de treinamento razoáveis, cuja qualidade parece bastante dependente de uma anotação bem feita. Os próximos passos em nosso trabalho envolvem a análise, de um ponto de vista estrutural das regras implementadas por Bikel em seu \emph{parser}.

Uma das principais alterações que tivemos que fazer foi a construção das \emph{head-find rules} para os constituintes encontrados no corpus utilizado, primeiro porque as TAGS utilizadas pelo \emph{Treebank}, não são as mesmas e segundo porque as regras de formação dos constituintes para o português são diferentes do inglês. Notamos que as \emph{head-find rules} são de grande importância para o desempenho do \emph{parser} de Bikel que implementa o modelo gerativo baseado na noção \emph{head-centering}, em que o núcleo é o elemento principal de todo o processo de geração.

O \emph{Parser} de Bikel, além de ser possível emular os modelos de Collins, também possibilita diferentes parametrizações quanto a algoritmos utilizados e implementação de regras de \emph{head-find}, por exemplo. Alguns parâmetros melhoram a performance quanto a acertos no momento de análise de sentenças mas levam um tempo maior para finalizar, já outras configurações permitem que o tempo de análise das sentenças seja mais rápido porém a qualidade do resultado diminui consideravelmente.



