Dan Bikel desenvolveu uma ferramenta de \emph{parsing} estatístico extensível que permite diferentes tipos de configuração de modelos estatísticos e gerativos, incluindo emulação dos modelos de \emph{parsing} de Michael Collins com performance semelhante. Pode ser facilmente estendida para novos domínios e novas linguagens. 

Baseada no modelo 2 de Collins \cite{collins99}, permite uma grande gama de parametrizações, implementa e estende o modelo de análise, que inclue análise lexicalizada orientada ao núcleo das sentenças, modelo que incorpora diferentes níveis de informações estruturais, que ja foram descritas anteriormente nesse trabalho. 

O analisador permite extenções específicas. Além de usar o pacote em inglês para determinar uma linha de base para análise de precisão, foi criado um pacote par ao português. Este pacote fornece regras para descoberta dos núcleos dos sintágmas, tratamento especial para quando os heads são explicitamente marcados, características morfológicas, e algumas opções de ajuste do analisador ao Floresta.

Modelos baseados no núcleo dos sintágmas deve permitir saber quem é o filho do núcleo anterior durante o processo de treino, Esta informação não é codificada no formato PTB, para o inglês o pacote fornece uma série de heurísticas para inferir o núcleo do sintágma.

A ferramenta de Dan Bikel faz uso de parâmetros que definem particularidades do seu funcionamento, sendo alterados dependendo de sua aplicação. Os parâmetros não só permitem alterar alguns comportamentos como também ``\emph{plugar}'' classes modificadas para serem usadas no lugar das classes padrões, para fazer tratamento especifico do processamento de linguagem definidos algoritmicamente.

As regras de identificação dos núcleos dos sintagmas também são configuráveis e utiliza-se para isto um arquivo de parâmetros de \emph{head-rules}, necessário para a implementação dos modelos baseados na noção \emph{head-centering}, em que o núcleo é o elemento principal e direcionador de todo o processo de geração de uma árvore sintática como falado anteriormente.

Os parâmetros estão separados em dois arquivos, arquivo de parâmetros e arquivo de regras de \emph{head-rules}, e são utilizados no momento de treinamento do \emph{parser} e no momento de analisar as frases. Os principais parâmetros são:

\section{Parâmetros de utilização do \emph{parser} de Dan Bikel}
\label{sec:bikelest_param}

\scriptsize

\textbf{O Conjunto de parâmetros abaixo permite com que o \emph{parser} de Bikel Emule o Modelo 2 definito por Michael Collins.}\\

\textbf{parser.language=english}\\
Especifica o idioma que será analisado.

\textbf{parser.language.package=danbikel.parser.english}\\
Especifica o pacote referente ao idioma analisado

\textbf{parser.language.wordFeatures=danbikel.parser.english.SimpleWordFeatures}\\
Especifica o nome da classe que estende WordFeatures no pacote da língua.

\textbf{parser.downcaseWords=false}\\
Especifica se as palavras devem ser convertidas para minúsculas durante o treino e decodificação.

\textbf{parser.subcatFactoryClass=danbikel.parser.SubcatBagFactory}\\
Especifica qual subclasse de SubcatFactory deve ser utilizada como instanciador.

\textbf{parser.shifterClass=danbikel.parser.BaseNPAwareShifter}\\
Especifica qual classe deve ser utilizada como \emph{Shifter}.

\textbf{parser.language.training=portuguese.NPArgThreadTraining}\\
Especifica qual classe que implementa a interface Training deve ser utilizada para efetuar o treinamento.

\HRule \\

\textbf{Parâmetros para classe danbikel.parser.Model}\\

\textbf{parser.model.precomputeProbabilities=true}\\
A propriedade especifica se deve ou não pré-calcular probabilidades no treino e utilização desses probabilidades pré-computada na decodificação.

\textbf{parser.model.collinsDeficientEstimation=true}\\
A propriedade especifica se deve ou não fazer estimativa deficiente de probabilidades, como o bug descrito na tese de Michael Collins.

\textbf{parser.model.prevModMapperClass=danbikel.parser.Collins}\\
Especifica qual classe que implementa a interface NonterminalMapper deve ser utilizada pelo NTMapper para mapear não-terminais que são modificadores previamente gerados de algum não-terminal \emph{head}.

\textbf{parser.model.doPruning=true}\\
A propriedade especifica se os parâmetros redundantes de cada instancia da classe Model devem ser removidos.

\textbf{parser.model.pruningThreshold=0.05}\\
A propriedade especifica um fator de quando o \emph{pruning} deve ser realizado.

\HRule \\

\textbf{Parâmetros para Modelos de Bikel, mas é ignorado quando o parâmetro danbikel.model.precomputeProbabilities é ajustado como true}\\

\textbf{parser.modelCollection.writeCanonicalEvents=true}\\
A propriedade indica ou não se a classe ModelCollection deverá salvar o (grande) \emph{HashMap} contendo as versões canônicas das instancias de \emph{Event} quando é serializado em disco. Ao decodificar usando caches ao invés de probabilidades pré-computadas (ver precomputeProbs), a criação da tabela de eventos canônicos economiza tempo, deixando o decodificador salvar no cache os eventos observados durante o treino ao invés de sempre ter que criar os eventos canônicos, dinamicamente, durante a decodificação.

\HRule \\

\textbf{Parâmetros necessários para o treinamento do parser}\\

\textbf{parser.training.addGapInfo=false}\\
Propriedade para especificar se Training.addGapInformation(Sexp) adiciona a informação de \emph{gap} ou deixa a formação da árvores intocadas.

\textbf{parser.training.collinsRelabelHeadChildrenAsArgs=true}\\
A propriedade especifica se o Training.identifyArguments(Sexp) deve re-rotular as os nodo filhos que são \emph{head} como argumentos. Essa rerotulação é desnecessária, uma vez que os \emph{heads} já são inerentemente distintos dos outros filhos, mas é realizada (e possivelmente um bug) no \emph{parser} de Collins e, por isso, está disponível como uma configuração aqui, a fim de simular o mesmo modelo.

\textbf{parser.training.collinsRepairBaseNPs=true}\\
A propriedade especifica se Training.repairBaseNPs(Sexp) altera a estrutura da árvore ou a deixa intacta.

\HRule \\

\textbf{Parâmetros para classe danbikel.parser.Trainer}\\

\textbf{parser.trainer.unknownWordThreshold=6}\\
A propriedade especifica o limite de ocorrência em que abaixo as palavras são consideradas desconhecidos pelo treinador.

\textbf{parser.trainer.countThreshold=1}\\
A propriedade especifica o limite em que abaixo as instancias de TrainerEvent são descartadas pelo treinador.

\textbf{parser.trainer.reportingInterval=1000}\\
A propriedade especifica o intervalo (em número de períodos) em que o treinador emite relatórios para System.err enquanto treina.

\textbf{parser.trainer.numPrevMods=1}\\
A propriedade especifica quantos modificadores anteriores a instancia de \emph{Trainer} deve gerar saída.

\textbf{parser.trainer.numPrevWords=1}\\
A propriedade especifica quantos núcleos (\emph{heads}) dos modificadores anteriores a instancia de \emph{Trainer} deve gerar saída.

\textbf{parser.trainer.keepAllWords=true}\\
A propriedade especifica se a instancia de \emph{Trainer} deve manter registro de todas as palavras. Normalmente, as palavras abaixo de um limite de ocorrência são mapeados como desconhecidas.

\textbf{parser.trainer.keepLowFreqTags=true}\\
A propriedade especifica se a instancia de \emph{Trainer} inclui palavras de baixa freqüência no seu mapa de POS.

\textbf{parser.trainer.collinsSkipWSJSentences=true}\\
A propriedade especifica se algumas frases são ignoradas durante o treino, a fim de emular o \emph{parser} de Michael Collins. Essa opção só deve ser utilizada com o \emph{Penn Treebank Wall Street Journal}.

\HRule \\

\textbf{Parâmetros para a classe danbikel.parser.Decoder}\\

\textbf{parser.decoder.useLowFreqTags=true}\\
A propriedade especifica se deve utilizar \emph{tags} coletadas de palavras de baixa frequência pelo treinador.

\textbf{parser.decoder.useCellLimit=false}\\
A propriedade especifica se o decodificador deve impor um limite no número de itens por célula na tabela.

\textbf{parser.decoder.cellLimit=10}\\
A propriedade especifica o limite para o número de itens que o decodificador terá por célula. Este tipo de poda só irá ocorrer se decoderUseCellLimit está ativado.

\textbf{parser.decoder.usePruneFactor=true}\\
A propriedade indica se o algoritmo deve podar ou não as arvores geradas dentro de um determinado fator.

\textbf{parser.decoder.pruneFactor=4}\\
A propriedade para especificar o fator pelo qual o decodificador deverá podar as arvores geradas. 

\textbf{parser.decoder.useCommaConstraint=true}\\
A propriedade  especifica se o decodificador deve empregar restrições sobre como vírgulas podem aparecer segundo uma regra de CFG.
$Z --> <.. X Y..>$

\textbf{parser.decoder.useHeadToParentMap=true}\\
A propriedade para especificar se o decodificador deve usar o mapeamento de nodos \emph{heads} para seus pais, derivados durante o treino.

\textbf{parser.decoder.useSimpleModNonterminalMap=true}\\
Este é mecanismo pelo qual o decodificador tenta calcular a probabilidade de um não-terminal ser modificador no contexto de um nodo pai e um núcleo.

Exemplo:\\
Se existe um NP a esquerda de um VP, cujo nodo pai é um S durante o treinamento, então o modificador não-terminal iria conter o mapeamento S, VP, left --> NP.

\textbf{Parâmetros de configuração do pacote, substitua <língua> pelo nome do pacote}\\
\textbf{parser.wordfeatures.<língua>.useUnderscores=true}\\
Propriedade que define se o símbolo ``\_'' será incluído ou não no vetor de caracteres.

\textbf{parser.headtable.<língua>=data/head-rules.lisp}\\
Define onde estão as regras de determinação dos núcleos.

\HRule \\

%\scriptsize
%\setlongtables
%\begin{longtable}{|p{3.5cm}|p{7.5cm}|p{5cm}|}
%\caption{Parâmetros do parser de Dan Bikel} \\
%\hline
%Parâmetro & Descrição & Valores válidos \\
%\hline
%\endhead
%\hline
%\endfoot
%parser.language & Especifica o idioma que será usado com a configuração & Nome do idioma, exemplo: ``portuguese'' \\
%\hline
%parser.language.package & O nome do pacote Java que será usado contendo as classes especificas para a lingua definida & Nome do pacote java, exemplo: ``parser.portuguese'' \\
%\label{tab:parâmetros}
%\end{longtable}




\section{Formato do arquivo de parâmetros}
\label{sec:bikel_formato_aqrquivo}


\scriptsize
\begin{verbatim}

#     WordNet Parser
#    Settings to emulate Mike Collins' 1997 Model 2
#
parser.language=english
parser.language.package=danbikel.parser.english
parser.language.wordFeatures=danbikel.parser.english.SimpleWordFeatures
parser.downcaseWords=false
parser.subcatFactoryClass=danbikel.parser.SubcatBagFactory
parser.shifterClass=danbikel.parser.BaseNPAwareShifter
#
# settings for danbikel.parser.Model
parser.model.precomputeProbabilities=true
parser.model.collinsDeficientEstimation=true
parser.model.prevModMapperClass=danbikel.parser.Collins
#
# settings for danbikel.parser.ModelCollection
#    the following property is ignored when
#    danbikel.model.precomputeProbabilities is true
parser.modelCollection.writeCanonicalEvents=true
#
# settings for danbikel.parser.Training
parser.training.addGapInfo=false
parser.training.collinsRelabelHeadChildrenAsArgs=true
parser.training.collinsRepairBaseNPs=true
#
# settings for danbikel.parser.Trainer
parser.trainer.unknownWordThreshold=6
parser.trainer.countThreshold=1
parser.trainer.reportingInterval=1000
parser.trainer.numPrevMods=1
parser.trainer.numPrevWords=1
parser.trainer.keepAllWords=true
parser.trainer.keepLowFreqTags=true
parser.trainer.globalModelStructureNumber=1
parser.trainer.collinsSkipWSJSentences=true
parser.trainer.modNonterminalModelStructureNumber=2
parser.trainer.modWordModelStructureNumber=2
#
# settings for danbikel.parser.CKYChart
parser.chart.itemClass=danbikel.parser.CKYItem$MappedPrevModBaseNPAware
parser.chart.collinsNPPruneHack=true
#
# settings for danbikel.parser.Decoder
parser.decoder.useLowFreqTags=true
parser.decoder.useCellLimit=false
parser.decoder.cellLimit=10
parser.decoder.usePruneFactor=true
parser.decoder.pruneFactor=4
parser.decoder.useCommaConstraint=true
parser.decoder.useHeadToParentMap=true
parser.decoder.useSimpleModNonterminalMap=true
#
#
# settings specific to language package danbikel.parser.english
#
parser.wordfeatures.english.useUnderscores=true
parser.headtable.english=data/head-rules.lisp
parser.training.metadata.english=data/training-metadata.lisp
\end{verbatim}

\normalsize

\section{Formato do arquivo de \emph{head-find rules}}
\label{sec:bikel_formato_aqrquivo_head_find}


O arquivo de \emph{head-rules} contém as regras que determinam a construção das árvores sintáticas necessárias, definindo as sentenças e seu núcleo sintático.

Abaixo segue um exemplo de definição das \emph{head-rules} utilizadas em testes de utilização do \emph{parser}.

\textbf{Regras de \emph{head-find} para anotação do Bosque}

\scriptsize

\begin{verbatim}

(
(NP (l N PROP PRONPERS PRONINDP NADJ NP))
(VP (l VFIN VINF VPCP VGER) (l VP))
(ADJP (l ADJ ADJP) (l PRONDET))
(ADVP (r ADV ADVP))
(CU (r CONJC CU , ;))
(X (l VP))
(PP (l PRP PP))
(FCL (l VP) (l NP))
(ICL (l VP) (l NP))
(ACL (l VP) (l NP))
(* (l))
)
\end{verbatim}

\normalsize

As regras definidas acima são essenciais para que no momento de treino o \emph{parser} defina exatamente a qual classe pertence as palavras classificando-as de maneira correta sintaticamente.

Por exemplo, a regra (VP (l VFIN VINF VPCP VGER) (l VP)) define que o núcleo deve ser o primeiro nodo filho, da esquerda para direita, que seja um VFIN, VINF, VPCP ou VGER. Caso não exista, então o núcleo será o mesmo do primeiro elemento, da esquerda para a direita, que seja um VP.

Serão experimentadas algumas alterações nos valores de alguns parâmetros acima citados para avaliação dos seus efeitos, e analisaremos os resultados obtidos no capitulo posterior.

