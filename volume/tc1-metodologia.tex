Este trabalho possui um forte componente experimental e exploratório. Assim, em termos metodológicos, a cada experiência realizada, os resultados obtidos devem ser analisados quantitativa e qualitativamente, para orientar as correções nos parâmetros do \emph{parser} ou indicar a necessidade de alterações como: a) de pré-processamento ou pós-processamento dos casos; ou b) no código do \emph{parser}. Nesse sentido, a avaliação quantitativa é um componente importante e será feita de forma rigorosa. Pretende-se utilizar os métodos de avaliação tradicionais de \emph{precision/recall} \cite{black93} e possivelmente outras a definir.

Será usado um sistema de controle de versão que permite que se trabalhe com diversas versões dos arquivos de trabalho
e  versões do software durante nossos testes e implementações.

Foi dada uma maior atenção quando ao tratamento das palavras desconhecidas pois acreditamos que um ajuste nesse ponto se pode alcançar desempenhos melhores.

Para trabalhar com o corpus no formato de entrada para o parser de Bikel foi necessário pré-processamento para eliminar ruídos e criar dados de entrada no formato PTB, mesmos obstáculos encontrados por Baldridge \cite{baldridge06} e Bonfante\cite{bonfante03} em seus trabalhos. 

A presença de ruído nos dados do corpus é inevitável pois a maioria das sentenças são anotadas de maneira automática, e algumas inconsistências ainda foram encontradas no decorrer do trabalho. Ruídos são possívelmente provenientes da complexidade na construções da língua que pode gerar ambiguidade ou estruturas complexas, o que dificulta a análise (em alguns casos dificulta até a análise humana). %De um modo geral, é possível separar as incorreções encontradas, decorrentes da análise automática, basicamente em dois grupos:

%- aquele em que as incorreções são fruto de algum tipo de ambiguidade na classificação morfológica e/ou sintática das palavras, e que são, de alguma forma simples de resolver com a intervenção humana;

%- e outro em que as incorreções são fruto de estruturas complexas da língua, de resolução difícil até para o revisor humano (tais casos requerem discussão e opinião de todo o grupo envolvido no projeto).

Não foi feita alteração quanto as TAGS utilizadas no corpus de trabalho, apenas quanto a pontuação de hífen, que será sibstituida por Underline.

Como experimento mais avançado foram feitos testes utilizado lematização das palavras do corpus, primeiro com todas as palavras das sentenças, logo apenas com os verbos. A fim de verificar a qualidade do treino executado pela ferramenta de Bikel.

Nos primeiros experimentos foram feitas apenas alterações nas regras de \emph{head-find}. O Corpus de treino, desenvolvimento e teste utilizado é o Bosque da Floresta Sintática que contém um total de 5200 sentenças separadas em 4160 para treino 520 para desenvolvimento e 520 para teste, nas primeiras baterias de teste utilizamos pouco menos de 100 sentenças para ajustar parâmetros mais rapidamente, na última foi utilizado 520 sentenças na fase de desenvolvimento e teste.

Na execução dos experimentos inicialmente trabalhamos sobre a influêcia da escolha do núcleo \emph{(Head)} dos constituíntes que é essencial na implementação dos algorítmos de implementação dos modelos propostos por Collins \cite{collins99}.

O corpus do Bosque ja tem anotados muitos dos \emph{heads} das sentenças disponibilizadas, porém o parser de bikel precisa de todos os \emph{heads}, e verificou-se que nem todas as sentenças possuem essa informação, tendo que ser analisada e construída de forma empírica.

A ferramenta de Bikel possibilita a utilização de um módulo para especificação de regras para determinar o \emph{head} de um constituinte e optou-se por se usar esse módulo e construir regras baseadas nas regras de formação da lingua portuguesa incrementalmente (já exemplificado em capítulo anterior). 

Primeiramente trabalhamos com a configuração default para o inglês, com parâmetros e head-find rules definidas para a gramática definida no PTB. A segunda configuração de testes usamos as configurações para o português definidas por Dan Bikel em seu parser e com as regras de head-find rules básicas, ou seja, primeiro o núcleo da sentença é a primeira palavra da direita para a esquerda, regra essa que é padrão na lingua inglesa, logo se alterou para que o núcleo da sentença seja a primeira palavra da esquerda para a direita, regra que ja se aproxima da regra de formação da lingua portuguesa. Após testes com essas configurações iniciais, trabalhou-se incrementalmente na evolução tanto de parametros melhores do parser quanto a definição das head-find rules para a lingua portuguesa.

É interessante mensionar que a construção das regras foi incremental e experimentos foram sendo feitos para avaliar a qualidade das regras criadas como ilistrado nas figuras ....

O último conjunto de regras utilizado é o que estamos usando atualmente, bem melhor que o original conforme resultados apresentados. Acreditamos que não teremos ganho incremento nos resultados tentando melhorar as regras de head-find, e sim se ajustarmos melhor os parâmetros da ferramenta.

A segunda sequência de experimentos tera como objetivo avaliar a granularidade das POS tags no corpus utilizado. ....

...


A ideia básica é que tags devem ser distintos quando a categorias tem distribuições sintáticas diferentes, por outro lado se duas classes tem mesma distribuição ou distribuição próxima, separa-las apenas levará a perda de qualidade quanto a informação sintática constante nas sentenças ...,  ...

Verificou-se conforme esperado que para verbos é bastante relevante as sub-categorias pois a distribuição sintática das diferentes categorias verbais é bem distinta, da mesma forma para pronomes porque os possessivos tem diferente distribuição que os pessoais; os .. tem posição de pré-modificadores nominal e os outros ...

Diferenciando conjunçao coordenada e subordinativa nao houve diferença significativa nos resultados.

O último caso relativo a nomes refere-se  o que ... do corpus 

