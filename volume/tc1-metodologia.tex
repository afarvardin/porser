Este trabalho possui um forte componente experimental. Assim, em termos metodológicos, a cada experiência realizada, os resultados obtidos devem ser analisados quantitativa e qualitativamente, para orientar as correções nos parâmetros do \emph{parser} ou indicar a necessidade de alterações como: a) de pré-processamento ou pós-processamento dos casos; ou b) no código do \emph{parser}. Nesse sentido, a avaliação quantitativa é um componente importante e será feita de forma rigorosa. Pretende-se utilizar as metodologias tradicionais de \emph{precision/recall} \cite{black93} e possivelmente outras a definir.

Será usado um sistema de controle de versão que permite que se trabalhe com diversas versões dos arquivos de trabalho
e  versões do software durante nossos testes e implementações.



