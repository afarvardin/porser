A linguagem é um dos aspectos mais fundamentais do comportamento humano e um componente crucial de nossas vidas. A linguagem em sua modalidade escrita serve como um registro a longo prazo do conhecimento que é passado de geração em geração. Na forma falada, ela serve diariamente como meio primário de coordenação do nosso comportamento e interatividade entre as pessoas.

A linguagem  é estudada em diferentes meios ou áreas acadêmicas. Cada disciplina define os seus próprios problemas e possui seus próprios métodos para resolvê-los. A linguística, por exemplo, estuda a estrutura da própria linguagem, considerando perguntas como: a) por que certas combinações de palavras compõem uma sentença e outras não; ou b) por que algumas sentenças possuem significado e outras não. A psicolinguística estuda o processo de produção e compreensão da língua pelos seres humanos, levando em consideração perguntas como: a) como as pessoas escolhem, ou identificam, a estrutura apropriada das frases; ou b) como elas decidem os significados para cada palavra.

O objetivo da linguística computacional é utilizar os conhecimentos desenvolvidos nas áreas citadas - e outras relacionadas - para desenvolver teorias e aplicações utilizando as noções de algoritmos e estrutura de dados para processar e entender sintática e semanticamente a língua, escrita ou falada.

Nas últimas duas décadas, o desenvolvimento de métodos estatísticos para o processamento de linguagem natural (PLN) vem sendo impulsionado pela evolução no poder de processamento dos computadores \cite{manning99, jurafsky}. Esses métodos utilizam grande quantidade de dados, em conjunto com cálculos probabilísticos, para tentar ``entender'' corretamente a estrutura e o significado da linguagem. Fundamental nesse processo foi o surgimento de extensos \emph{corpora} sintaticamente anotados (\emph{treebanks}) \cite{marcus93,marcus94,abeille03,sardinha04}.

Neste trabalho focou-se o processo de \emph{parsing} probabilístico baseado em \emph{corpus}, em particular, utilizando como base os estudos e o \emph{parser} desenvolvido por Michael Collins \cite{collins99,collins97}, na sua versão posterior reimplementada por Dan Bikel \cite{bikel04}, sendo construído um \emph{parser} para a lingua portuguesa.

%Este trabalho pretende estudar o processo estatístico de \emph{parsing} baseado em \emph{corpus} aplicado à língua portuguesa. Iremos focar o processo de \emph{parsing}, mais especificamente as abordagens estatísticas baseadas em \emph{corpus}, para solucionar esse problema e desenvolver, especificamente, o processo de \emph{parsing} probabilístico aplicado à língua portuguesa. Iremos utilizar como base os estudos e o \emph{parser} desenvolvido por Michael Collins \cite{collins99,collins97} na versão posterior reimplementada por Dan Bikel \cite{bikel04}.


