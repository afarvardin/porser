Este trabalho está organizado em cinco capítulos. Inicialmente neste primeiro capítulo, introduzimos o tema do trabalho, apresentando a motivação e os objetivos. No segundo capítulo apresentam-se o referencial teórico envolvido e trabalhos anteriores que serviram de referência na comparação de resultados. Logo após no terceiro capítulo apresentamos um aprofundamento teórico das técnicas estudadas e conceitos matemáticos implementadas na ferramenta de \emph{parser} utilizada neste trabalho. No capítulo quatro descrevemos a metodologia utilizada, os métodos de avaliação propriamente ditos, e de que maneira os experimentos foram conduzidos, descrevendo os testes realizados juntamente com seus respectivos resultados e finalmente no quinto capítulo são elaboradas algumas considerações finais e possíveis trabalhos futuros.
